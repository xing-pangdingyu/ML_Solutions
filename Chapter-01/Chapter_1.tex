\documentclass[12pt,reqno]{amsart}
\usepackage{amsmath,amsfonts,amscd,amssymb,epsf,color,enumerate,graphicx,url}
\setlength{\oddsidemargin}{-0.2in}%
\setlength{\evensidemargin}{-0.2in}%
\setlength{\textwidth}{6.6in}%
\setlength{\topmargin}{-0.5in}%
 \setlength{\textheight}{9.5in}%
 \definecolor{orange}{rgb}{1,0.5,0}
 \pagestyle{plain}
\linespread{1.3}
\usepackage[small]{caption}
\newcommand{\pa}{\partial}
\newcommand{\va}{\vspace{0.4cm}}
\newcommand{\di}{\displaystyle}
\newcommand{\disp}{\displaystyle}


% turn on \answertrue to show the solution
% turn on \answerfalse to hide the solution
\newif\ifanswer
\answertrue
%\answerfalse



\begin{document}
\noindent {\footnotesize Hands-On Machine Learning}\hspace{10.5cm} {\footnotesize Solutions}

\vspace{0.5cm}
\hspace{5.5cm}\textbf{\large Exercises in Chapter 1}
\vspace{0.5cm}

\begin{enumerate}[1.]

\item How would you define machine learning?

\ifanswer
\noindent {\bf Solution}
\begin{enumerate}[(i)]
\item The field of study that gives computers the ability to learn without being explicitly programmed.
\item A computer program is said to learn from experience $E$ with respect to some task $T$ and some performance measure $P$,
if its performance on $T$, as measured by $P$, improves with experience $E$.
\end{enumerate}
\vspace{1cm}



\item Can you name four types of applications where it shines?

\ifanswer
\noindent {\bf Solution}
\begin{enumerate}[(i)]
\item Detecting scamming emails.
\item Detecting credit card fraud.
\item Summarizing long documents automatically.
\item Classifying images of products.
\end{enumerate}
\vspace{1cm}



\item What is a labeled training set?

\ifanswer
\noindent {\bf Solution}
A labeled training set is a training set where the training data are labeled, that is,
we tell the machine how they are classified.
\vspace{1cm}



\item What are the two most common supervised tasks?

\ifanswer
\noindent {\bf Solution}
Classification and Predicting numeric value.
\vspace{1cm}



\item Can you name four common unsupervised tasks?

\ifanswer
\noindent {\bf Solution}
Clustering algorithm, visualization algorithm, anomaly detection and association rule learning.
\vspace{1cm}



\item What type of algorithm would you use to allow a robot to walk in various unknown terrains?

\ifanswer
\noindent {\bf Solution}
Reinforcement learning.
\vspace{1cm}



\item What type of algorithm would you use to segment your customers into multiple groups?

\ifanswer
\noindent {\bf Solution}
Unsupervised learning.
\vspace{1cm}



\item Would you frame the problem of spam detection as a supervised learning problem or an unsupervised learning problem?

\ifanswer
\noindent {\bf Solution}
Supervised learning problem.
\vspace{1cm}



\item What is an online learning system?

\ifanswer
\noindent {\bf Solution}
An online learning system is a system that you train incrementally by feeding data instances sequentially,
either individually or in small groups called \textit{mini-batches}.
\vspace{1cm}



\item What is out-of-core learning?

\ifanswer
\noindent {\bf Solution}
Out-of-core learning means to use online learning algorithms to train models on huge datasets that cannot fit in one machine's main memory.
The algorithm loads part of the data, runs a training step on that data, and repeats the process until it has run on all of the data.
\vspace{1cm}



\item What type of algorithm relies on a similarity measure to make predictions?

\ifanswer
\noindent {\bf Solution}
Instance-based learning.
\vspace{1cm}



\item What is the difference between a model parameter and a model hyperparameter?

\ifanswer
\noindent {\bf Solution}
\begin{enumerate}[(i)]
\item A model parameter is a parameter that effects directly on the model.
\item A model hyperparameter is a parameter of a learning algorithm,
and is applied as the amount of regularization during learning.
\end{enumerate}
\vspace{1cm}



\item What do model-based algorithms search for? What is the most common strategy they use to succeed? How do they make predictions?

\ifanswer
\noindent {\bf Solution}
\begin{enumerate}[(i)]
\item A Model-based algorithm is used to build a model of the examples and to make predictions with the model.
\item The strategy is to minimize the cost function that measures how bad the model is.
\item Given a new instance, the model uses the features of the new instance and the model parameters to make the prediction.
\end{enumerate}
\vspace{1cm}



\item Can you name four of the main challenges in machine learning?

\ifanswer
\noindent {\bf Solution}
Insufficient challenges, nonrepresentative training data, overfitting the training data and
underfitting the training data.
\vspace{1cm}



\item If your model performs great on the training data but generalizes poorly to new instances, what is happening? Can you name three possible solutions?

\ifanswer
\noindent {\bf Solution}
\begin{enumerate}[(i)]
\item The training data are not representative of the new instances the model predicts on.
\item The training data is overfit.
\item The amount of training data is insufficient.
\end{enumerate}
\vspace{1cm}



\item What is a test set, and why would you want to use it?

\ifanswer
\noindent {\bf Solution}
\begin{enumerate}[(i)]
\item The test set is a subset of our data where we test the model that is trained using the training set.
\item We use the test set to get an estimate generalization error of our model on new cases.
\end{enumerate}
\vspace{1cm}



\item What is the purpose of a validation set?

\ifanswer
\noindent {\bf Solution}
The validation set is used when we need to test multiple models and hyperparameters.
If we test all hyperparameters in the test set, we will get the best model and hyperparameter
for the particular test set, but not for new cases in production. So, we split a validation
set and use it to get the best performed model and hyperparameter, before testing using the
test set.
\vspace{1cm}



\item What is the train-dev set, when do you need it, and how do you need it?

\ifanswer
\noindent {\bf Solution}
The train-sev set is a subset of the training set. It is used when we cannot recognize it is because our model has overfit the
training set, or it is due to the mismatch between the training data and cases in production.
After the model is trained on the training set (exclude the train-dev set), we evaluate it on the
train-dev set. If the model performs bad, we know the problem is the overfitting the training data. Otherwise,
we test it on the validation set, and if it performs bad, the problem is the mismatch.
\vspace{1cm}



\item What can go wrong if you tune hyperparameters using the test set?

\ifanswer
\noindent {\bf Solution}
As said in Exercise 17, if we tune hyperparameters using the test set, the hyperparameter
works best particularly for the test set, but it is likely that it will not work well for new
cases in production.
\vspace{1cm}






\end{enumerate}

\end{document}
